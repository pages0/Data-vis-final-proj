\documentclass[12pt]{article}

\usepackage{latexsym,amssymb,amsthm,amsmath,amscd,multirow,marvosym,fixltx2e,tikzsymbols}

\usepackage{tikz,latexsym,amssymb,amsthm,amsmath,amscd,multirow,marvosym,mathrsfs,fixltx2e}
\usetikzlibrary{svg.path}


\setlength{\oddsidemargin}{0cm}
\setlength{\evensidemargin}{0cm}
\setlength{\textwidth}{6.5in}
\setlength{\headheight}{-.5 in}
\setlength{\topmargin}{0in}
\setlength{\textheight}{9 in}



\begin{document}

\begin{center} 
  CSC 395 Spring 2017 \\
  Report\\
  by Connor Gregorich-Trevor, Nick Roberson, and Reilly Noonan Grant\\
  Due: May 12th \\

----------------------------------------------------------------------------------------------------------------------- \end{center}

\begin{enumerate}
  \item[Introduction] 
    
    Our goal was to visualize the relationship between stock price and social media interest 
    in various companies. We later scaled this back to examining the relationship between 
    stock price and ``search interest'', for which we used Yahoo Stocks and Google trends. 
    We did this by creating a plot which simultaneously graphed the google trend data and
    historical stock price for a company, then added the ability to switch between companies
    displayed. We decided that it would be helpful if users could see events which occured 
    around the same time shifts in stock price or search interest, so we added functionality 
    so that if one clicks on a data point, the visualization displays New York Times articles
    related to the company that appeared in the month leading up to the date of the data point.
    We also added functionality so that when one of these data points is moused over 
    the legend will display the exact data that the data point contains. Over the course of
    implementing this project, we learned that the javascript date library can be difficult 
    to work with, that good data can be hard to find, and that the relationship between 
    stock price and social media interest can often be difficult to fully understand.
    
    
  \item[Motivation] 
    As social media becomes increasingly important in the world, and companies care more about 
    their presence on social media, understanding the impact that popular attention has on companies 
    becomes more important. By helping users visualize the relationship between search trends
    and stock prices, we allow them to explore the effect of social media presence. Being able to 
    explore in this way is particularly important, as the relationship between stock price and social
    media data is not always easy to understand. Social media attention can be positive or negative
    for a company, depending on the context. Because of this, the combination of a graph 
    and related New York Times headlines allows users to more easily to understand the historical 
    context of shifts in stock, or popularity. With this graph, the user can answer not only 
    questions such as ''Does the price of stock generally increase when social media attention increases,"
    but also ''What events led up to such a drastic reduction in stock price and increase in social media attention for Delta Airlines in 2017?''

  \item[Results]


  \item[Reflection]
  
  \item[Conclusion]


\end{enumerate}
\end{document}